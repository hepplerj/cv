\section{Professional Experience}\label{professional-experience}

\textbf{George Mason University}, Fairfax, VA

\quad Senior Developer, Roy Rosenzweig Center for History and New Media,
2021--\emph{present}

\quad Adjunct Faculty, Department of History and Art History,
2024--\emph{present}

\vspace{.4cm}

\textbf{University of Nebraska at Omaha}, Omaha, NE

\quad Digital Engagement Librarian, UNO Libraries, 2017--2021

\quad Assistant Professor of History (by courtesy), 2017--2021

\quad Subject Specialist, History, UNO Libraries, 2019--2021

\quad Sustainability Faculty, Sustainability Minor Program, 2018--2021

\quad Interim Institutional Repository Coordinator, 2021

\vspace{.4cm}

\textbf{Stanford University}, Stanford, CA

\quad Academic Technology Specialist, Department of History and
\href{http://cidr.stanford.edu}{Center for Interdisciplinary Digital
Research} (CIDR), 2013--2017

\quad Research staff, \href{http://cesta.stanford.edu}{Center for
Spatial and Textual Analysis} (CESTA), 2013--2017

\quad Lecturer, Department of History, 2013--2016

\vspace{.4cm}

\textbf{\href{https://endangereddataweek.org}{Endangered Data Week}}

\quad Co-founder and organizer, 2017--2021

\vspace{0.2cm}

\subsubsection{Major Affiliations and
Fellowships}\label{major-affiliations-and-fellowships}

\textbf{Affiliate Fellow}, Center for Great Plains Studies, University
of Nebraska-Lincoln, Lincoln, NE (2023--\emph{present})

\textbf{Affiliate Scholar}, Connecting the Interstates, NEH ODH Planning
Grant (2020--\emph{present})

\textbf{Member}, Environmental History Action Collaborative Working
Group, Environmental Data \& Governance Initiative
(2020--\emph{present})

\textbf{Mentor}, Open Leadership Program, Mozilla Foundation
(2018--2020)

\newpage

\subsubsection{Editorial and Advisory
Appointments}\label{editorial-and-advisory-appointments}

\textbf{Advisory Board}, Open Online Newspaper Initiative
(2022--\emph{present})

\textbf{Advisory Board}, Sourcery (2021--\emph{present})

\textbf{Digital Content Advisory Board},
\emph{\href{http://www.americanyawp.com/}{The American Yawp}}
(2014--\emph{present})

\textbf{Editorial Board}, \emph{Studies in Midwestern History},
Midwestern History Association (2017--2021)

\textbf{Editorial Board},
\emph{\href{https://uimiddle.wordpress.com/}{The Middle West Review}},
University of Nebraska Press (2014--2021)

\section{Areas of Research and
Teaching}\label{areas-of-research-and-teaching}

North American West and Great Plains \(\cdot\) Twentieth-century United
States \(\cdot\) Digital History \(\cdot\) Environmental History
\(\cdot\) Data Visualization \(\cdot\) Public History \(\cdot\) Software
Development

\section{Education}\label{education}

\textbf{University of Nebraska-Lincoln}, Ph.D., Department of History,
August 2016\\
\emph{Dissertation}:
``\href{http://digitalcommons.unl.edu/historydiss/86/}{Machines in the
Valley: Community, Urban Change, and Environmental Politics in Silicon
Valley, 1945--1990}.'' \emph{Advisors}: Patrick Jones (chair), William
G. Thomas, Margaret Jacobs, James Garza, Stephen Ramsay

\textbf{University of Nebraska-Lincoln}, M.A., Department of History,
May 2009\\
\emph{Thesis}:
``\href{http://digitalcommons.unl.edu/historydiss/21/}{Framing Red
Power: The Trail of Broken Treaties, the American Indian Movement, and
the Politics of Media}.'' \emph{Advisors}: John Wunder (chair), Douglas
Seefeldt, Thomas Gannon

\textbf{South Dakota State University}, B.A., History, May 2007\\
\emph{Undergraduate Thesis}: ``\,`We Lit a Fire Across Indian Country':
The American Indian Movement and the Intellectual Origins of Red
Power.'' \emph{Minor}: Economics

\section{Research}\label{research}

\subsubsection{Books}\label{books}

\emph{Silicon Valley and the Environmental Inequalities of High-Tech
Urbanism}. Environment in Modern North America series. Norman:
University of Oklahoma Press, 2024.

\emph{\href{https://ucincinnatipress.manifoldapp.org/projects/digital-community-engagement}{Digital
Community Engagement: Partnering Communities with the Academy}}.
Co-editor with Rebecca Wingo and Paul Schadewald. Cincinnati: University
of Cincinnati Press, 2020.\\
\quad \(\cdot\) Awarded the National Council on Public History (NCPH)
Book Award, 2021

\subsubsection{Peer-Reviewed Journal
Articles}\label{peer-reviewed-journal-articles}

Heidi Blackburn and Jason A. Heppler, ``Hidden Voices: A Case Study
Analysis of Subject Headings for Book Titles on Women in Science,''
\emph{Science and Technology Libraries}, (2022).
\url{https://doi.org/10.1080/0194262X.2022.2040405}

Jeanne Reames, Jason A. Heppler, and Cory Starman, ``Becoming
Macedonian: Name Mapping and Ethnic Identity. The Case of Hjphaistion.''
\emph{Karanos. Bulletin of Ancient Macedonian Studies} (2020), 3, 11-37.

Heidi Blackburn and Jason A. Heppler, ``Who is Writing about Women in
STEM in Higher Education? A Citation Analysis of Gendered Authorship,''
\emph{Frontiers in Psychology} 10 (2020).
\url{https://doi.org/10.3389/fpsyg.2019.02979}

Heidi Blackburn and Jason A. Heppler, ``Women in STEM in Higher
Education: A Citation Analysis of the Current Literature,''
\emph{Science \& Technology Libraries} (2019).
\url{https://doi.org/10.1080/0194262X.2019.1645080}

Brandon Locke and Jason A. Heppler, ``Teaching Data Literacy for Civic
Engagement: Resources for Data Capture and Organization,'' \emph{KULA:
Knowledge Creation, Dissemination, and Preservation Studies} 2 (1) (Fall
2018). \url{https://doi.org/10.5334/kula.23}

``Green Dreams, Toxic Legacies: Toward a Digital Political Ecology of
Silicon Valley,'' \emph{International Journal of Humanities and Arts
Computing} vol.~11, no. 1 (March 2017): 68--85.
\url{https://doi.org/10.3366/ijhac.2017.0179}

Gabriel Wolfenstein and Jason A. Heppler,
``\href{http://tah.oah.org/content/crowdsourcing-digital-public-history/}{Crowdsourcing
Public Digital History},'' \emph{The American Historian}, March 2015.

Alex Galarza, Douglas Seefeldt, and Jason A. Heppler,
``\href{http://journalofdigitalhumanities.org/1-4/a-call-to-redefine-historical-scholarship-in-the-digital-turn/}{A
Call to Redefine Historical Scholarship in the Digital Turn},''
\emph{Journal of Digital Humanities}, December 2012.

\subsubsection{Articles}\label{articles}

``\href{https://www.washingtonpost.com/outlook/2019/04/26/how-silicon-valley-provides-blueprint-cleaning-up-our-drinking-water/}{How
Silicon Valley provides the blueprint for cleaning up our drinking
water},'' \emph{The Washington Post}, April 26, 2019.

``Advocacy, Training, and Awareness Through Endangered Data Week,''
co-author with Brandon Locke, Sarah Melton, and Rachel Mattson,
\emph{Parameters}, SSRC, February 26, 2018.

``\href{https://theconversation.com/how-silicon-valley-industry-polluted-the-sylvan-california-dream-85810}{How
Silicon Valley Industry Polluted the Sylvan California Dream},''
\emph{The Conversation}, November 15, 2017.

\subsubsection{Book Chapters}\label{book-chapters}

``Creating Capacity for Research Data Services at Regional Universities:
A Case Study,'' co-author with Omer Farooq and Kate Ehrig-Page, in
\emph{\href{https://www.alastore.ala.org/trdm}{Teaching Research Data
Management}}, ed.~Julia Bauder. ALA Editions, 2022.

``A National Monument,'' co-author with Douglas Seefeldt, in \emph{The
Companion to Custer and the Little Big Horn}, ed.~Brad Lookingbill.
Hoboken: Wiley and Sons, 2015.

``The American Indian Movement and South Dakota Politics,'' in \emph{The
Plains Political Tradition}, ed.~Jon Lauck, John E. Miller, and Donald
Simmons. Pierre, SD: South Dakota State Historical Society Press, 2011,
267--287.

\subsubsection{Other Academic
Publications}\label{other-academic-publications}

Molly Taylor-Polesky and Jason Heppler, ``The 1613 Marriage Journey of
Elizabeth Stuart: Reflections on Visualizing European Geopolitics on the
Brink of the Thirty Years War,'' Spatial History Project, Stanford
University, June 20, 2018:
\url{https://web.stanford.edu/group/spatialhistory/cgi-bin/site/pub.php?id=129}.

Arguing with Digital History working group, ``Digital History and
Argument,'' white paper, Roy Rosenzweig Center for History and New
Media, George Mason University, November 13, 2017:
\url{http://rrchnm.org/argument-white-paper/}.

\subsubsection{Electronic Books}\label{electronic-books}

\emph{\href{http://hepplerj.github.io/rubyist-historian/}{The Rubyist
Historian: Ruby Fundamentals for Humanities Scholars}}. DOI:
10.5281/zendo.9987.

\subsubsection{Digital History}\label{digital-history}

Project Director,
\emph{\href{http://dissertation.jasonheppler.org}{Machines in the
Valley: Growth, Conflict, and Environmental Politics in Silicon
Valley}}, 2015--2017.

Project Manager, \emph{\href{http://codyarchive.org/}{William F. Cody
Archive}}, Center for Digital Research in the Humanities, University of
Nebraska-Lincoln, 2011--2013.

Project Director,
\emph{\href{http://www.codystudies.org/showindians/}{``Self-sustaining
and a good citizen'': William F. Cody and the Progressive Wild West}},
Cody Studies, Buffalo Bill Center of the West, 2012.

Software Developer and Scholarly Contributor,
\emph{\href{http://buffalobillproject.unl.edu/}{The Buffalo Bill
Project}}, Department of History, University of Nebraska-Lincoln,
2011--2013 (retired and superseded by
\href{http://www.codystudies.org/}{Cody Studies}).

Project Director, \emph{Buffalo Bill's Wild West and the Progressive
Image of American Indians}, digital history, 2010--2011 (retired and
superseded by \href{http://www.codystudies.org/showindians/}{William F.
Cody and the Progressive Wild West}).

Graduate Digital Editor, \emph{The Papers of William F. Cody}, Buffalo
Bill Historical Center, 2009--2010.

Project Director, \emph{\href{http://framingredpower.org}{Framing Red
Power: The Trail of Broken Treaties, the American Indian Movement, and
the Politics of Media}}, digital history, 2008--2009.

\subsubsection{Collaborative Digital
Projects}\label{collaborative-digital-projects}

\emph{Connecting Threads}, Roy Rosenzweig Center for History and New
Media, George Mason University (in progress, 2022--\emph{present}):
https://connectingthreads.co.uk. (co-PI)

\emph{Death by Numbers}, Roy Rosenzweig Center for History and New
Media, George Mason University (in progress, 2021--\emph{present}):
https://deathbynumbers.org. (senior developer)

\emph{American Religious Ecologies}, Roy Rosenzweig Center for History
and New Media, George Mason University (in progress,
2018--\emph{present}): https://religiousecologies.org. (senior
developer)

\emph{Collecting These Times: American Jewish Experiences of the
Pandemic}, Roy Rosenzweig Center for History and New Media, George Mason
University (in progress, 2020--2022): https://collectingthesetimes.org.
(senior developer)

\emph{\href{http://aidhp.com}{American Indian Digital History Project}}
(2017--\emph{present}): http://aidhp.com. (research director)

\emph{\href{https://omahahistorical.org}{Nebraska Historical}}
(2017--2020): https://omahahistorical.org. (project director)

\emph{\href{http://svhistorical.org}{Silicon Valley Historical}}
(2016--2020): http://svhistorical.org. (project director)

\subsubsection{Contributions to
Publications}\label{contributions-to-publications}

Author, \href{http://jasonheppler.org}{JasonHeppler.org},
2008--\emph{present}.

Contributor, \href{http://blogwest.org/}{BlogWest}, 2013--2020.

Contributor, \href{http://whadigitalfrontiers.com/}{Digital Frontiers: A
WHA Digital Workshop}, 2010--\emph{present}.

Contributor, \href{http://chronicle.com/blogs/profhacker/}{ProfHacker},
2012--2013.

Contributor,
\href{https://www.insidehighered.com/blogs/gradhacker}{GradHacker},
2012.

Contributor, \href{http://digitalhistory.unl.edu/}{Doing Digital
History}, 2009--2010.

\subsubsection{Media}\label{media}

Co-host, \href{http://overanalyze.fireside.fm}{Overanalyze}, podcast,
with Andy Wilson, 2016--2017.

Co-founder, \href{http://fiddly.fm}{FiddlyFM}, 2015--2017.

Co-host, \href{http://www.fiddly.fm/firstdraft/}{The First Draft
Podcast}, with Elijah Meeks and Paul Zenke, 2014--2016.

\subsubsection{Select Academic Software}\label{select-academic-software}

``Superfundr: An R data package on U.S. Superfund sites,'' R package
(2019--\emph{present}): \url{https://github.com/hepplerj/superfundr/}

\subsubsection{Reviews}\label{reviews}

Digital history review of Erika Bsumek,
\emph{\href{https://cliovis.com}{ClioVis: Visualizing Connections}}, in
\emph{The Journal of American History} (2024).

Review of Katie Phillips, ``Timeline of Empire,'' in \emph{Reviews in
Digital Humanities} (2022).
\url{https://reviewsindh.pubpub.org/pub/timeline-of-empire/release/3}

Review of Kathleen A. Cairns \emph{At Home In the World: California
Women and the Postwar Environmental Movement} (University of Nebraska
Press), in the \emph{Pacific Historical Review} (2022).
\url{https://doi.org/10.1093/whq/whac013}

Review of Trevor Owens, \emph{The Theory and Craft of Digital
Preservation} (Johns Hopkins University Press), in \emph{The Public
Historian} (May 2020).
\url{https://doi-org.leo.lib.unomaha.edu/10.1525/tph.2020.42.2.160}

Review of Walter Nugent, \emph{Color Coded: Party Politics in the
American West, 1950-2016} (University of Oklahoma Press), in \emph{South
Dakota History} (April 2020).

Review of Pamela Riney-Kehrberg, \emph{The Nature of Childhood: An
Environmental History of Growing Up in America since 1865} (University
Press of Kansas), in \emph{Middle West Review} (Spring 2019).
\url{https://doi.org/10.1353/mwr.2019.0030}

Review of Andrew Busch, \emph{City in a Garden: Environmental
Transformations and Racial Justice in Twentieth-Century Austin}
(University of North Carolina Press), in \emph{Environmental History}
(Fall 2018). \url{https://doi.org/10.1093/envhis/emy067}

Digital history review of \emph{Mapping Inequality: Redlining in New
Deal America} and \emph{Renewing Inequality: Family Displacements
through Urban Renewal}, in \emph{American Panorama: An Atlas of United
States History} (University of Richmond), in \emph{American Quarterly}
(Fall 2018). \url{http://doi.org/10.1353/aq.2018.0058}

Digital history review of Jay Taylor, \emph{Follow the Money: A Spatial
History of In-Lieu Programs for Western Federal Lands} (Stanford Spatial
History Project), in \emph{Western Historical Quarterly} (Fall 2018).
\url{https://doi.org/10.1093/whq/why049}

Digital history review of \emph{American Panorama: An Atlas of United
States History} (University of Richmond), in \emph{Journal of American
History} (March 2017). \url{https://doi.org/10.1093/jahist/jaw624}

Digital history review of \emph{North Dakota: People Living on the Land}
(State Historical Society of North Dakota), in \emph{The Public
Historian} (May 2017). \url{https://doi.org/10.1525/tph.2017.39.2.104}

Review of Bradley Shreve, \emph{Red Power Rising: The National Indian
Youth Council and the Origins of Native Activism} (University of
Oklahoma Press), in \emph{South Dakota History} (Spring 2012).

Digital tool review, ``Google Earth for Historians,'' in
\emph{\href{http://digitalhistory.unl.edu/t-reviews/geheppler.php}{The
Digital History Project}} (Summer 2010).

Review of Joseph Trimbach, \emph{American Indian Mafia: An FBI Agent's
True Story about Wounded Knee, Leonard Peltier, and the American Indian
Movement} (Outskirts Press), in \emph{South Dakota History} (Spring
2009).

Digital tool review, ``TokenX as a Historical Research Tool,'' with
Brent M. Rogers, in
\emph{\href{http://digitalhistory.unl.edu/t-reviews/tokenxhepplerrogers.php}{The
Digital History Project}} (July 2009).

Review of Akim Reinhardt, \emph{Ruling Pine Ridge: Oglala Lakota
Politics from the IRA to Wounded Knee}, in \emph{South Dakota History},
(Summer 2008).

Digital history review of William G. Thomas et al., ``The Countryside
Transformed: The Eastern Shore of Virginia, the Pennsylvania Railroad,
and the Creation of the Modern Landscape,'' in \emph{The Digital History
Project} (February 2008).

\section{Public History and Public Humanities
Experience}\label{public-history-and-public-humanities-experience}

\textbf{Research director}, \emph{American Indian Digital History
Project} (2016--\emph{present}). An Omeka exhibit of material on
American Indian history. Served as a technical advisor in deciding on
and setting up technical infrastructure, metadata design, and document
digitization. Primary investigator: Kent Blansett.

\textbf{Director}, \emph{\href{https://omahahistorical.org}{Nebraska
Historical}} (2018--2021). An online spatial exhibit on the history of
Nebraska.

\textbf{Technical lead}, \emph{Chinese Railroad Workers in North
America}, Stanford University (2013--2016). Served as a technical
advisor and developer for a metadata management system, helped organized
and curate nearly 10,000 digitized objects, and trained sixteen research
assistants on metadata best practices and the platform used for
collecting metadata. Primary investigators: Gordon Chang and Shelley
Fischer Fishkin.

\textbf{Project manager}, \emph{\href{http://codyarchive.org}{William F.
Cody Archive}}, Center for Digital Research in the Humanities,
University of Nebraska-Lincoln (2011--2013). Supervised two graduate
student interns, maintained documentation on project milestones and
reference material, worked closely with museum staff at the Buffalo Bill
Center for the West, and helped organize nearly 24,000 digitized
documents.

\textbf{Museum aide}, South Dakota Art Museum, South Dakota State
University (2006--2007). Worked with museum curator in preparing
exhibits, maintaining an inventory of museum objects, and interacting
with museum patrons.

\section{Community Outreach and
Engagement}\label{community-outreach-and-engagement}

\textbf{Consultant}, \emph{\href{http://invisiblehistory.ops.org}{Making
Invisible Histories Visible}}, Omaha, Nebraska (2017--2021). This
program invites twenty-four high school freshmen from Omaha Public
Schools to take part in a summer program focused on the history of their
city. I serve as a consultant primarily on mapping and
digitally-interactive aspects of the projects produced in the program.

\textbf{Consultant}, Calgary Stampede, Calgary, Alberta, Canada (2019).
Provided feedback and advice on migrating from PastPerfect to Omeka for
collections management and suggestions on possible alternatives for
Omeka for working with community volunteers.

\textbf{Project Lead}, Puerto Rico Hurricane Relief Mapathon, Omaha,
Nebraska (2017). I planned and hosted a mapathon to help with Puerto
Rico disaster relief, organizing an open source mapping session to
provide data to the Red Cross through OpenStreetMap.

\textbf{Organizer} and \textbf{Project Lead},
\emph{\href{https://mozilla.github.io/global-sprint/}{Mozilla Global
Sprint 2017}}, Omaha, Nebraska (2017). I coordinated and hosted Global
Sprint 2017 to bring together scientists, educators, artists, engineers,
and others to hack and build projects for a healthy internet. We focused
our project around \href{http://github.com/open-omaha/}{Open Omaha}.

\textbf{Project Lead}, \emph{Mobile Digitization for Rural Community
Archives}, Omaha, Nebraska (2017-2021). With a grant from the LYRASIS
Catalyst Fund, we initiated a pilot program for a mobile digitization
lab and maker space for working with rural community archives and Tribal
colleges.

\textbf{Founding Member and Organizer},
\emph{\href{http://endangereddataweek.org}{Endangered Data Week}},
Omaha, Nebraska (2017--\emph{present}). Helped organize the first
international Endangered Data Week, a collaborative effort among
campuses, nonprofits, libraries, citizen science initiatives, and
cultural heritage institutions. EDW is designed to shed light on public
datasets that are in danger of being deleted, repressed, mishandled, or
lost. Our sponsors included the Digital Library Federation and their
Records Transparency and Accountability Interest Group, the Council on
Library and Information Resources, the National Digital Stewardship
Alliance, Data Refuge, and the Mozilla Science Lab.

\textbf{Presenter}, \emph{Information Exchange}, Omaha, Nebraska (2017).
An event devoted to helping communities locate and strategize methods to
provide better access to information, generate ideas about what
libraries are and how knowledge can be shared within a community to
improve the lives of its citizens.

\textbf{Consulting scholar}, \emph{Santa Clara Valley National Heritage
Area} (2016--2017). Research historian helping aid the pursuit of the
National Park Service designation for recognizing Silicon Valley as a
National Heritage Area (NHA), only the second such site in California.

\section{Invited Talks and
Presentations}\label{invited-talks-and-presentations}

Keynote address, ``Digital History and Communities in an Age of
Surveillance,'' Institute for Digital Research in the Humanities (IDRH),
University of Kansas, March 7, 2024.

Public Historian Speaker Series, University of Montana, Missoula,
Montana, April 4, 2023.

``Visualizing Data in the Liberal Arts,'' Colorado State University,
April 4, 2019, Fort Collins, Colorado.

Panel, ``Shaping the Digital Humanities: UNL Alumni,'' Digital
Humanities Forum, University of Nebraska-Lincoln, April 2018, Lincoln,
Nebraska.

Keynote address, ``Approaching a New Historical Atlas of Midwestern
History with Deep Maps and Digital History'' Midwestern History
Association, June 2017, Grand Rapids, Michigan.

``\,`Don't Let Industry Do its Business In Our Water!!!': Urban Space
and Environmental Politics in Silicon Valley,'' Ball State University,
April 2017, Muncie, Indiana.

Talkback, \emph{The Man Who Shot Liberty Valance}, Omaha Community
Playhouse, February 2017, Omaha, Nebraska.

``Doing Digital History in the Classroom,'' Missouri Valley History
Conference, March 2015, Omaha, Nebraska.

``Humanistic Approaches to Data Visualization,''
\texttt{d3.digitalhumanities()} Meetup, January 2015, San Francisco,
California.

Teaching with Technology, Nebraska Wesleyan University, June 2012,
Lincoln, Nebraska.

History Club Graduate School Symposium, South Dakota State University,
October 2007, Brookings, South Dakota.

\section{Media Appearances}\label{media-appearances}

``Alternative Academic Careers,'' GradHacker Podcast, February 2013.

\section{Conference Papers, Talks, and
Workshops}\label{conference-papers-talks-and-workshops}

\subsubsection{Papers Presented}\label{papers-presented}

``Ranching Interests and Anti-Government Politics in the Sagebrush-Era
West,'' American Society for Environmental History, March 2024, Denver,
Colorado.

``Space, Time, and Visualizing the London Bills of Mortality,'' Southern
History Association, November 11, 2022, Baltimore, Maryland.

``Digitizing and Transcribing the 1926 Census of Religious Bodies,''
Society for the Scientific Study of Religion and the Religious Research
Association, November 11-13, 2022, Baltimore, Maryland.

``\,`As the people came, the orchards went': Water, Energy, and Land in
Silicon Valley, 1945-1970,'' Annual Meeting of the Agricultural History
Society, August 2022, Stavanger, Norway.

``Visualizing Silicon Valley's Urban Spaces,'' panel on
\emph{Visualizing the West: Digital History as Process and Product},
61st Annual Conference of the Western History Association, October 2021,
Portland, Oregon.

``Hidden Voices: A Data Analysis of Subject Headings for Books on Women
in Science,'' with Heidi Blackburn, American Women of Science:
Recovering History, Defining the Future 2020 Symposium, Smithsonian
American Women's History Initiative, October 2020, virtual.

``\,`Don't Let Industry Do its Business in our Water!!!' High-Tech
Toxics and Environmental Justice in Silicon Valley,'' American Society
for Environmental History, March 2020, Ottawa, Ontario, Canada.
(Cancelled due to the pandemic.)

``Using data visualization to analyze topic development by business
communication students in a one-shot setting,'' Brick \& Click: An
Academic Library Conference, November 2019, Maryville, Missouri.

``A citation analysis of journals publishing research on women in STEM
in higher education,'' American Library Association Annual Conference,
June 2019, Washington, D.C.

``Creating a Culture of Data Consciousness with Endangered Data Week,''
Mozilla Festival, October 2018, London, United Kingdom.

``\,`Piecemeal, Patchwork, Prop-Up': Urban Planning and Environmental
Change in Postwar Silicon Valley,'' at ``You Are Here'': An
Interdisciplinary Conference on Place, Space, and Embodiment, Creighton
University, March 2018, Omaha, Nebraska.

``The Tap Water Rebellion: Pollution, High-Tech Industrialization, and
Suburban Politics in Silicon Valley,'' at the American Society for
Environmental History, March 2018, Riverside, California (did not
attend).

``\emph{Oskate Wicasa}, Progressive Thought, and the Digital Publics of
Buffalo Bill's Wild West,'' 57th Annual Western Historical Association,
November 2017, San Diego, California.

``Introducing the American Indian Digital History Project,'' at the 57th
Annual Western Historical Association, November 2017, San Diego,
California.

``Mobile Rural Archives Lab,'' LYRASIS Member Summit, October 2017,
Philadelphia, Pennsylvania.

``Mapping Silicon Valley'' Digital Humanities 2017, August 2017,
Montreal, Quebec, Canada.

``Digital History, Publics, and Community Engagement,''
\emph{Information Exchange}, March 2017, Omaha, Nebraska.

``Green Dreams, Toxic Legacies: Digital History and the Landscapes of
Silicon Valley,'' Annual Meeting of the American Society for
Environmental History, March 2016, Seattle, Washington (did not attend).

``From the Laptop to the Archive: Managing Research Data in the
Humanities,'' October 2015, Digital Library Federation Forum, October
2015, Vancouver, British Columbia, Canada.

``Mapping Silicon Valley's Environment: Deep Maps and Urban
Environmental History'' 55th Annual Conference of the Western History
Association, October 2015, Portland, Oregon.

``Minimal Computing in Digital Humanities,'' Academic Technology Expo,
Stanford University, October 2015.

``Networks in the Humanities,'' lighting talk, with Brian Sarnacki,
Rebecca Wingo, and Andy Wilson, HASTAC, May 2015.

``Mapping the American West through the U.S. Post,'' Social Science
History Association Annual Meeting, November 2014, Toronto, Canada.

``Digital Humanities in the Classroom,'' Academic Technology Expo,
Stanford University, October 2014, Stanford, California.

``Spatial History and the Western Past,'' Six Shooters Lighting Talk,
53rd Annual Conference of the Western History Association, October 2013,
Tucson, Arizona.

``Programming in the Humanities,'' proposed and led session, THATCamp
American Historical Association, January 2012, Chicago, Illinois.

``The American Indian Movement and South Dakota Politics,'' George
McGovern Conference, November 2011, George McGovern Center for
Leadership and Public Service, Mitchell, South Dakota.

``\,`This Great and Sovereign Right': Eminent Domain, the Cherokee
Nation, and Railroad Law, 1880--1890,'' 50th Annual Conference of the
Western History Association, October 2010, Incline Village, Nevada.

``William F. Cody and the Digital Frontier,'' International Cody Studies
Conference, August 2010, Cody, Wyoming.

``Framing Red Power: Newspapers and Native Activists during the Trail of
Broken Treaties,'' Digital Humanities Summit, March 2010, Lawrence,
Kansas.

``\,`Another Wounded Knee Was Feared': Visualizing Newspaper Narratives
of the Trail of Broken Treaties Caravan and American Indian Activism
with Digital Tools,'' 53rd Annual Missouri Valley History Conference,
March 2010, Omaha, Nebraska.

``Red Power and the Intellectual Origins of the American Indian
Movement,'' 50th Annual Missouri Valley History Conference, March 2007,
Omaha, Nebraska.

``Tangled Memories: Wounded Knee and the Problem of Memory,'' 41st
Annual Northern Great Plains History Conference, October 2006, Sioux
Falls, South Dakota.

``Tangled Memories: Wounded Knee and the Problem of Memory,'' 38th
Annual Dakota History Conference, Sioux Falls, South Dakota.

``Case Study of the Makah Tribe's Defense of Whaling Rights,'' 14th
Annual SDSU American Indian History and Culture Conference, Brookings,
South Dakota.

\subsubsection{Chair/Comment}\label{chaircomment}

Chair, '' Digital History Lightning Talks,'' Annual Conference of the
Western History Association, October 2017, November 2018, October 2019.

Chair and comment, ``The Digital Second Crusade,'' The Second Crusade:
Chronicles and Perspectives Undergraduate Student Conference, University
of Nebraska at Omaha, April 2017, Omaha, Nebraska.

Panel Comment, ``Integrating Digital History into Research \& Teaching:
The American Experience in World War II,'' Missouri Valley History
Conference, March 2017, Omaha, Nebraska.

Panel chair, ``Sources as Data: Opportunities and Challenges,'' 131st
Annual Meeting of the American Historical Association, January 2017,
Denver, Colorado.

Panel chair, ``Western Encounters, Alliances, and Experiences: Mormons,
Indians, and U.S. Federal Law,'' 54th Annual Conference of the Western
History Association, October 2014, Newport Beach, California.

Discussion facilitator, ``Breakout Session: Games and Learning,''
Academic Technology Expo, Stanford University, October 2014, Stanford,
California.

Panel chair, ``Exploring Historic Landscapes with Mobile Technology,''
53rd Annual Conference of the Western History Association, October 2013,
Tucson, Arizona.

Panel chair, Fourth Annual James A. Rawley Conference in the Humanities,
University of Nebraska-Lincoln, April 2009.

Panel chair, Third Annual James A. Rawley Conference in the Humanities,
University of Nebraska-Lincoln, April 2008.

\subsubsection{Roundtables}\label{roundtables}

``History and the Spatial Turn: Pedagogical Approaches,'' round table
discussant, at the 136th American Historical Association, January 2023,
Philadelphia, Pennsylvania.

``Uncovering the Work of Women in Science in Library, Archive, and
Museum Collections,'' American Women of Science: Recovering History,
Defining the Future 2020 Symposium, Smithsonian American Women's History
Initiative, October 2020, virtual.

``Housing, Belonging, and Displacement in the 20th-Century West,'' 60th
Annual Conference of the Western History Association, October 2020,
virtual.

``It's a Gamble: Luck, Chance, and Failure as a Historian,'' round table
discussant, 59th Annual Conference of the Western History Association,
October 2019, Las Vegas, Nevada.

``Passages from Quantitative History to Digital Humanities,'' round
table discussant, at the 133rd American Historical Association, January
2019, Chicago, Illinois.

``Arguing with Digital History,'' roundtable discussant, at the 132nd
American Historical Association, January 2018, Washington, D.C.

``Digital History Roundtable,'' roundtable discussant, 150th Annual
Meeting of the American Historical Association, January 2016, Atlanta,
Georgia (did not attend).

``The Graduation Threshold: A Discussion on Bridging the Gap from
Graduation to Employment,'' co-organizer and roundtable discussant with
Elaine Nelson, Leisl Carr-Childers, Sarah Keyes, and Elise Boxer, 55th
Annual Conference of the Western History Association, October 2015,
Portland, Oregon.

``Digital Scholarship, Academic Careers, and Tenure,'' roundtable with
Jana Remy, Andrew Torget, Mills Kelly, and Katina Rogers, 129th Annual
Meeting of the American Historical Association, January 2015, New York
City, New York.

``Digital Pedagogy for History: Lightning Round,'' roundtable, 129th
Annual Meeting of the American Historical Association, January 2015, New
York City, New York.

``The Digital History Seminar,'' round table, with Douglas Seefeldt,
Brent Rogers, and Michelle Tiedje, 126th Annual Meeting of the American
Historical Association, January 2012, Chicago, Illinois.

``Historical Scholarship in the Digital Age: Asking New Questions with
Digital Technologies,'' roundtable discussion, 51st Annual Conference of
the Western Social Sciences Association, April 2009, Albuquerque, New
Mexico.

``Sandoz's Environment: \emph{Old Jules} and `The Vine','' roundtable,
5th Annual Mari Sandoz Heritage Society Conference, March 2008, Chadron,
Nebraska.

\subsubsection{Working Groups/Workshops}\label{working-groupsworkshops}

``Data Visualization for History,'' Institute for Digital Research in
the Humanities (IDRH), University of Kansas, March 7, 2024.

Attendee, IIIF 2023 Annual Conference, June 2023, Naples, Italy.

``The Basics of Static Sites,'' Roy Rosenzweig Center for History and
New Media, George Mason University, November 2022, Fairfax, Virginia.

``Mapping Urban Planning and Environmental History in Silicon Valley,''
Spatial History Working Group, Center for History and New Media, George
Mason University, November 2021, Fairfax, Virginia.

Attendee, National Council on Public History Annual Meeting, March 2021,
virtual.

Attendee, \emph{Data Practices Conference}, Stanford University, October
2020, Stanford, California (virtual).

Attendee, \emph{Practicing Pedagogies Summer Teaching Summit},
University of Nebraska-Lincoln, May 2020, Lincoln, Nebraska (virtual).

Attendee, Nebraska Library Association's College \& University Section
and Technical Services Round Table, May 2020, virtual.

Attendee, CSV Conf, May 2020, virtual.

Attendee, Innovation in Pedagogy and Technology Symposium, May 2020,
Lincoln, Nebraska (virtual).

Attendee, \emph{Railroads in Native America: Reflections on the 150th
Anniversary of the Transcontinental Construction}, University of
Nebraska at Omaha, September 2019, Omaha, Nebraska.

Attendee, \emph{Practicing Pedagogies Summer Teaching Summit},
University of Nebraska-Lincoln, May 2019, Lincoln, Nebraska.

Attendee, Innovation in Pedagogy and Technology Symposium, May 2019,
Lincoln, Nebraska.

\emph{Dynamic Digital Methods for Integrating Local History into Public
History Institutions and the K-16 Classroom}, one-day workshop given at
the 58th Annual Western Historical Association, sponsored by the WHA
Committee on Teaching and Public Education, St.~Mary's University,
October 2018, San Antonio, Texas.

Attendee, \emph{Practicing Pedagogies Summer Teaching Summit},
University of Nebraska-Lincoln, May 2018, Lincoln, Nebraska.

Co-convenor, \emph{Digital Community Engagement Symposium}, Macalester
College, May 2018, St.~Paul, Minnesota.

\emph{Networks and Network Visualization}, one-day workshop given at the
132nd American Historical Association, January 2018, Washington, D.C.

``Arguing with Digital History,'' invited attendee for two-day workshop,
Roy Rosenzweig Center for History and New Media, George Mason
University, September 2017, Arlington, Virginia.

Attendee, Innovation in Pedagogy and Technology Symposium, May 2017,
Lincoln, Nebraska.

\emph{Mapping and Network Analysis Workshop}, Digital Scholarship Lab,
Ball State University, April 12, 2017, Muncie, Indiana.

Attendee, Midwest Archives Conference, April 2017, Omaha, Nebraska.

Attendee, Digital Humanities Forum, University of Nebraska-Lincoln,
April 2017, Lincoln, Nebraska.

\emph{Digital Drop-In Sessions}, participant, 131st American Historical
Association, January 2017, Denver, Colorado.

Attendee, \emph{Flat Places, Deep Identities: Mapping Nebraska and the
Great Plains}, Great Plains Symposium, March 2017, Lincoln, Nebraska.

\emph{Digital History Table Talks}, round table discussant, 131st
American Historical Association, January 2017, Denver, Colorado.

\emph{Networks and Network Visualization}, one-day workshop given at the
131st American Historical Association, January 2017, Denver, Colorado.

\emph{Research Tools for your Thesis}, one-day workshop taught at
Stanford University for the Bing Honors College Program, September 2016,
Stanford, California.

\emph{Digital Humanities Workshop}, two-day workshop taught at Colorado
State University for the Carl A. Bimson Humanities Seminar, July 26--27,
2016, Fort Collins, Colorado.

``Getting Started with Omeka,'' \emph{Digital Humanities Skills
Workshop}, Center for Spatial and Textual Analysis, Stanford University,
January 2016, Stanford, California.

\emph{Data Preparation and Data Uncertainty}, one-day workshop given at
the 130th Annual Meeting of the American Historical Association, January
2016, Atlanta, Georgia (did not attend).

Working group facilitator, ``Public History as Digital History as Public
History,'' National Council on Public History Annual Meeting, Nashville,
Tennessee, April 2015.

Attendee, mediaX Games and Learning Conference, Stanford University, May
2014, Stanford, California.

\emph{Digital History Workshop}, Space, Materials, and Media Conference,
Stanford University, September 2013.

\emph{Introduction to GIS}, session co-leader with Katie McDonough,
THATCamp Alt-Ac, University of California-Berkeley, Berkeley,
California, October 2013.

\emph{Introduction to Gephi}, workshop co-leader with Ryan Cordell,
Digital Humanities Summer Institute, Victoria, British Columbia, June
2013.

Attendee, THATCamp American Historical Association, January 2012,
Chicago, Illinois.

``Improving Writing in the Survey Course,'' with William G. Thomas and
Leslie Working, Teaching History Forum, Department of History,
University of Nebraska-Lincoln, October 2009.

\subsubsection{Posters}\label{posters}

``American Religious Ecologies,'' with Caroline Greer, Lincoln Mullen,
and John Turner, at the 136th American Historical Association, January
2023, Philadelphia, Pennsylvania.

``Hidden Voices: A Data Analysis of Subject Headings for Books on Women
in STEM,'' with Heidi Blackburn, Association of College \& Research
Libraries Annual (ACRL) Meeting 2021, April 2021, virtual.

``Using Maps and Metadata to Teach the History of Silicon Valley,''
Third Annual Academic Technology Expo, Stanford University, October
2016, Stanford, California.

Electronic poster session, ``Mapping Silicon Valley's
Environmentalism,'' 130th Annual Meeting of the American Historical
Association, January 2016, Atlanta, Georgia. (did not attend)

``The William F. Cody Archive,'' electronic poster session, Center for
Great Plains Studies Symposium, March 2012, Lincoln, Nebraska.

\section{Grants and Fellowships}\label{grants-and-fellowships}

\subsubsection{External (\$42,650)}\label{external-42650}

Public Engagement with Historical Records, National Historical
Publications \& Records Commission, National Archives, not awarded.

Toward an Open Monograph Ecosystem (TOME), 2019, awarded, \$15,000.

Humanities Nebraska Grant, American Indian Digital History Project,
2019, not awarded.

Library of Congress Teaching with Primary Sources Grant, 2018, awarded,
\$19,750.

``Mobile Archives for Rural Communities,'' LYRASIS Catalyst Fund,
University of Nebraska at Omaha, 2017, awarded, \$5,500.

Digital Humanities Summer Institute Tuition Scholarship, University of
Victoria, British Columbia, Canada, 2014, awarded, \$800.

Digital Humanities Summer Institute Tuition Scholarship, University of
Victoria, British Columbia, Canada, 2013, awarded, \$800.

Nebraska Humanities Council Mini Grant, James A. Rawley Conference,
University of Nebraska-Lincoln, 2010, awarded, \$800.

\subsubsection{Internal (\$9,105)}\label{internal-9105}

Civic Participation Grant, University of Nebraska at Omaha, 2020,
\$1,000.

Eugene and Sunny Thomas Fund, University of Nebraska at Omaha Libraries,
University of Nebraska at Omaha, 2018, \$5,500.

Faculty Research Grant, Charles W. \& Mary Caldwell Martin Fund,
University of Nebraska at Omaha, 2017, \$725.

Office of International Affairs Seed Grant, Stanford University, 2014,
not awarded.

John F. Stover Fellowship, Department of History, University of
Nebraska-Lincoln, 2012, awarded, \$400.

Plains Humanities Alliance Grant, James A. Rawley Conference, University
of Nebraska-Lincoln, 2010, awarded, \$500.

Sheldon Travel Fund Award, Department of History, University of
Nebraska-Lincoln, 2010, awarded, \$600.

Sheldon Fund Award, Department of History, University of
Nebraska-Lincoln, 2010, awarded, \$380.

Addison E. Sheldon Fellowship, Department of History, University of
Nebraska-Lincoln, 2009, funded.

August Edgren Fellowship, College of Arts and Sciences, University of
Nebraska-Lincoln, 2007, funded.

\section{Awards and Honors}\label{awards-and-honors}

\begin{itemize}
\tightlist
\item
  External Funding Award Recipient and First Time Award, First Annual
  Faculty Research Awards, Office of Research and Creative Activity,
  University of Nebraska at Omaha, February 2018.
\item
  University Libraries Influence Award, University of Nebraska-Lincoln
  Libraries, 2011.
\item
  Dov Ospovat Memorial Award for Distinguished Graduate Research Paper,
  University of Nebraska-Lincoln, 2009.
\item
  Schultz-Werth Award for Outstanding Undergraduate Research Paper,
  South Dakota State University, 2007.
\item
  Phi Alpha Theta Honor Society, South Dakota State University,
  2005--2007.
\end{itemize}

\section{Professional Development}\label{professional-development}

RRCHNM--C\textsuperscript{2}DH Education in Digital History Summit, Roy
Rosenzweig Center for History and New Media and the Luxembourg Centre
for Contemporary and Digital History, April 6, 2022.

Attendee, ``Meville Nelles Hoffman Lecture in Environmental History,''
York University, January 13, 2022.

RRCHNM--C\textsuperscript{2}DH Education in Digital History Summit, Roy
Rosenzweig Center for History and New Media and the Luxembourg Centre
for Contemporary and Digital History, December 9 \& 16, 2021.

\emph{Equity in Action: Fostering an Antiracist Library Culture},
Library Journal, February 23-March 9, 2021.

``Digital Humanities and Spatial History: Atlantic World Stories,'' by
Zephyr Frank, UCL Centre for Digital Humanities, February 23, 2021.

\emph{The Power of Allies for an Inclusive Culture}, American Management
Association, February 18, 2021.

\emph{Introduction to Norwegian II}, University of Oslo (UiO), via
Future Learn. Completed 2020.

\emph{Indigenous Canada}, University of Alberta. Completed 2020.

\emph{Elements of AI}, University of Helsinki. Completed 2020.

``How do we respond to anti-Black racism in urbanist practices and
conversations?,'' Canadian Urban Institute, June 10, 2020.

\emph{Keep Teaching Summer Series}, Office of Digital Learning,
University of Nebraska at Omaha, weekly webinar series, June 24--August
4, 2020.

\emph{COVID-19 Contact Tracing}, Johns Hopkins University, via Coursera.
Completed 2020.

\emph{Python for Data Science}, University of California-San Diego, via
edX. Completed 2020.

\emph{Research Data Management and Sharing}, The University of North
Carolina at Chapel Hill, via Coursera. Completed 2020.

\emph{Norwegian for Beginners I}, Norwegian University of Science and
Technology (NTNU), via Future Learn. Completed 2020.

\emph{Introduction to Norwegian I}, University of Oslo (UiO), via Future
Learn. Completed 2020.

``Text Mining with HathiTrust: An Introduction for Librarians,'' May 4,
2020.

Red Hat Summit, sessions on AI, data, code, diversity and inclusion, and
analytics, virtual event, April 28-29, 2020.

``Solve Climate by 2030,'' Nebraska Power Dialog, April 12, 2020.

``Going Remote Q\&A Livestreams,'' Basecamp, March 24-April 2, 2020.

``Fighting the Climate Crisis,'' National Resource Defense Council,
teleconference, December 17, 2019.

\emph{Service Learning Seminar}, Service Learning Academy, University of
Nebraska at Omaha, July 2018.

\emph{Mozilla Open Leadership Framework Coaching}, Mozilla Foundation,
February--June 2018.

\emph{Mozilla Open Leaders Training: Mentorship and Training on Working
Open}, Mozilla Foundation, February--June 2018.

``Starting a Discussion (Leading to Action),'' Humanities Symposium,
University of Nebraska at Omaha, February 2018.

\emph{Python for Data Science and Machine Learning Bootcamp}, taught by
Jose Portilla, via Udemy. Completed 2017.

\emph{Data Science and Machine Learning Bootcamp with R}, taught by Jose
Portilla, via Udemy. Completed 2017.

\emph{The Data Scientist's Toolbox}, John Hopkins University, via
Coursera. Completed 2017.

\emph{Statistical Thinking for Data Science and Analytics}, Columbia
University Data Science Institute, via EdX. Completed 2017.

``Historical Trauma: Conquest, Genocide, and John Wayne, and WE are
still here,'' Carolyn Fiscus, Tribal Court Improvement Program, Omaha,
Nebraska, September 2017.

``Collections as Data: IMPACT,'' Library of Congress, July 2017.

``Open Leadership 101,'' Mozilla Foundation, 2017.

``Pre-Tenure Faculty Leadership Forum,'' University of Nebraska at
Omaha, 2017--2021.

Computation + Journalism Symposium, Stanford University, September 2016.

UseR! Conference, Stanford University, June 2016.

Workshop, ``CartoDB@Stanford,'' Stanford University, March 2015.

Participant, ``Desktop Fabrication and Physical Computing,'' Digital
Humanities Summer Institute, University of Victoria, British Columbia,
Canada, June 2014.

Workshop, ``Networks in History,'' Center for Spatial and Textual
Analysis, Stanford University, May 2014.

Workshop, ``See, Think, Design, Produce,'' Jonathan Corum, Bret Victor,
Mike Bostock, and Edward Tufte, San Jose, California, May 2014.

Symposium, Hestia2, Stanford University, November 2013.

Workshop, ``Geo for Higher Education Summit,'' Google, Mountain View,
California, July 2013.

Participant, ``Geographic Information Systems,'' Digital Humanities
Summer Institute, University of Victoria, British Columbia, Canada, June
2013.

Symposium, ``Visualization and the Research Process,'' Center for
Spatial and Textual Analysis, Stanford University, May 2013.

Workshop, Edward Tufte Visualization Workshop, San Jose, California, May
2013.

Workshop, ``Visualizing Complexity and Uncertainty,'' Stanford
Humanities Center, Stanford University, 2013.

Workshop, ``Cutting Edge Geospatial Technologies with Google Tools,''
Stanford University, February 2013.

Workshop, ``Digital Frontiers: A Digital History Workshop,'' Western
History Association, Denver, Colorado, October 2012.

Workshop, ``Open Access,'' Graduate Student Association, University of
Nebraska-Lincoln, October 2011.

Workshop, ``Building Your C.V. and Utilizing Social Networking for Your
Academic Career,'' HGSA Workshop, September 2011.

Participant, Cody Camp, Cody, Wyoming, July 2011.

Session attendee, ``Project Management,'' University of Nebraska-Lincoln
Libraries, March 2011.

Workshop, ``Digital Frontiers: A Digital History Workshop,'' Western
History Association, Incline Village, Nevada, October 2010.

Workshop, Nebraska Digital Workshop, Center for Digital Research in the
Humanities, University of Nebraska-Lincoln, 2008, 2009, 2010, 2011.

Workshop, ``Mapmaking in the Humanities,'' remote participation,
Electronic Data Center, Emory University, May 2009.

Workshop, ``Introduction to XML,'' Brian Pytlik-Zillig, Center for
Digital Research in the Humanities, University of Nebraska-Lincoln,
October 2007.

\section{Teaching}\label{teaching}

\textbf{NEH Institute for Advanced Topics in the Digital Humanities}

\begin{itemize}
\tightlist
\item
  Digital Methods for Military History (June 2021), George Mason
  University
\end{itemize}

\textbf{Digital Humanities Summer Institute}

\begin{itemize}
\tightlist
\item
  R, Interactive Graphics, and Data Visualization for the Humanities,
  with Lincoln Mullen (June 2016), University of Victoria.
\end{itemize}

\textbf{George Mason University (2021--\emph{present})}

\begin{itemize}
\tightlist
\item
  HIST 694: Digital Public History (Spring 2024)
\item
  SpaceCamp, spatial history workshop (Spring 2022)
\item
  Various workshops with the Data \& Spatial History Working Group,
  2021--\emph{present}
\end{itemize}

\textbf{University of Nebraska at Omaha (2017-2021)}

\emph{Courses}

\begin{itemize}
\tightlist
\item
  HIST 4900: Introduction to Digital Humanities independent study
  (Spring 2018) (3 credits)
\end{itemize}

\emph{Library Instruction}

\begin{itemize}
\tightlist
\item
  BootcampR: An Introduction to R (Spring 2020)
\item
  Guest lecture for ISQA 4000/8086: From Data to Decisions (Fall 2019)
\item
  Guest lecture for HIST 4990: Senior Research Seminar (Spring 2019)
\item
  Guest lecture for HIST 4740: Comparative Genocide (Spring 2019)
\item
  Guest lecture for HIST 1000: World Civilizations (Fall 2018)
\item
  Guest lecture for HIST 4910: Vietnam War (Fall 2018)
\item
  Guest lecture for HIST 4400: Native American History (Fall 2018)
\item
  Guest lecture for HIST 2980: Historical Methodology (Fall 2017, Spring
  2018, Fall 2018, Spring 2019, Fall 2019, Spring 2020, Fall 2020,
  Spring 2021)
\item
  Guest lecture for ENGL 3000: Louise Erdrich (Spring 2018)
\item
  Guest lecture for PA 3000: Applied Statistics and Data Management
  (Fall 2017)
\item
  Guest lecture for HIST 9100: Ancient Macedonia (Spring 2017)
\end{itemize}

\textbf{Stanford University (2013--2016)}

\begin{itemize}
\tightlist
\item
  Digital History Reading Group (2015--2016)
\item
  Digital History: Sources, Methods, Problems
  (\href{http://jasonheppler.org/teaching/hist205f.2014/}{Fall 2014}) (5
  credits)
\item
  \emph{Doing Digital History} workshop curriculum, Department of
  History (Winter 2013 and Fall 2013)
\end{itemize}

\textbf{University of Nebraska-Lincoln (2008--2010)}

\begin{itemize}
\tightlist
\item
  Teaching assistant, HIST 201: U.S. before 1877 (Prof.~Brenden Rensink,
  Fall 2010)
\item
  Teaching assistant, HIST 202: U.S. since 1877 (Prof.~William G.
  Thomas, Spring 2009)
\item
  Teaching assistant, HIST 150: African Culture and Civilization
  (Prof.~Dawne Curry, Fall 2008)
\end{itemize}

\subsubsection{Guest Lectures}\label{guest-lectures}

``Career Pathways for Historians,'' guest lecture, Digital Public
History, Prof.~Claire DuLaney and Prof.~Lori Schwartz, University of
Nebraska at Omaha, Fall 2023.

``Getting Started with Digital History,'' guest lecture, HIST 688:
History and Material Culture in the Digital Age, George Mason
University, Prof.~Deepthi Murali, Fall 2023.

``Mapping Silicon Valley's Environment,'' guest lecture, HIST 696: Clio
Wired, George Mason University, Prof.~Lincoln Mullen, Fall 2022.

``Introduction to Data Visualization,'' guest lecture, University of
Nebraska at Omaha, Lect. B. Shine Cho, Spring 2017.

``Digital Dissertations,'' discussion leader, Digital Humanities
Fellows, Center for Spatial and Textual Analysis, Stanford University,
Spring 2016.

``Spatial History,'' guest lecture, Stanford University, COMPLIST
238/ENGLISH 229D/HISTORY 229D: Introduction to Digital Humanities, Lect.
Brian Johnsrud and Mike Widner, Winter 2014.

``Mass Media and the American Indian Movement,'' guest lecture,
University of Nebraska-Lincoln, COMM 189H: Communications and Social
Movements, Prof.~Carly Woods, Fall 2011.

``Thomas Jefferson and the American West,'' guest lecture, University of
Nebraska-Lincoln, HIST 201: U.S. History before 1877, Prof.~Brenden
Rensink, Fall 2010.

``Empire of Liberty: Culture, Society, and Politics in the Early
Republic,'' guest lecture, University of Nebraska-Lincoln, HIST 201:
U.S. History before 1877, Prof.~Brenden Rensink, Fall 2010.

``Show Indians and Buffalo Bill's Wild West,'' guest lecture, University
of Nebraska-Lincoln, HIST 359: Mythic West, Prof.~Douglas Seefeldt,
Spring 2010.

``The Origins of Red Power and the American Indian Movement,'' guest
lecture, University of Nebraska-Lincoln, HIST 202: U.S. History after
1877, Prof.~William G. Thomas, Spring 2009.

\subsubsection{Workshops}\label{workshops}

``TextExpander,'' with Paul Zenke, Stanford Mobile Users Group, Stanford
University, October 2014, Stanford, California.

``1Password,'' with Vijoy Abraham and Paul Zenke, Stanford Mobile Users
Group, Stanford University, May 2014, Stanford, California.

``Research Management with Digital Tools,'' guest speaker, Prof.~Gordon
Chang, Stanford University, HIST 460: America in the World, February
2014.

``Using Technology in Teaching,'' discussion participant, Prof.~Nancy
Kollman, Stanford University, HIST 305: Workshop in Teaching History,
May 2013.

``Frameworks for Thinking about Teaching with Technology,'' discussion
participant, Stanford University, Center for Teaching and Learning, May
2013.

``Online Learning and History Education,'' discussion participant,
Stanford University, Center for Teaching and Learning, May 2013.

``Digital Frontiers: A Digital History Workshop,'' workshop co-leader,
Western History Association, Denver, Colorado, October 2012.

``Building Your C.V. and Utilizing Social Networking for Your Academic
Career,'' workshop co-leader, History Graduate Student Association
Workshop, September 2011.

``Introduction to XML/XSLT,'' workshop leader, University of
Nebraska-Lincoln, HIST 970: Digital History Seminar, Spring 2011.

\subsubsection{Discussion Leader}\label{discussion-leader}

Discussion leader, ``The Dust Bowl,'' Prof.~Douglas Seefeldt, HIST 352:
American West since 1900, Spring 2012.

Discussion leader, ``Buffalo Bill's Wild West,'' Prof.~Douglas Seefeldt,
HIST 352: American West since 1900, Spring 2011.

Discussion leader, Upton Sinclair's \emph{The Jungle}, Prof.~William G.
Thomas, HIST 202: U.S. After 1877, Spring 2009.

\subsubsection{Professional
Development}\label{professional-development-1}

Workshop, Graduate Teaching Assistant Workshop, Department of History,
University of Nebraska-Lincoln, August 2010.

\subsubsection{Advising}\label{advising}

\emph{Graduate Students}

\begin{itemize}
\tightlist
\item
  Rebecca Wall, African Colonial Employees Project, Stanford University,
  2014--2015
\item
  Liz Jacob, African Colonial Employees Project, Stanford University,
  2014--2015
\end{itemize}

\emph{Graduate Interns}

\begin{itemize}
\tightlist
\item
  J.P. Obley, William F. Cody Archive, University of Nebraska-Lincoln,
  2012
\item
  Andrew Tully, William F. Cody Archive, University of Nebraska-Lincoln,
  2012
\end{itemize}

\emph{Undergraduate Advising}

\begin{itemize}
\tightlist
\item
  Julia Laurence, Crowdsourcing Railroads, Stanford University, 2015
\item
  Rahul Singireddy, Chinese Input, Stanford University, 2015
\item
  KJ Fryauff, Follow the Money Project, Stanford University, 2015
\item
  KJ Fryauff, Chinese Railroad Workers Project, Stanford University,
  2014
\item
  Jocelyn Hickcox, Geography of the Post, Stanford University, 2013
\item
  Tara Balakrishnan, Geography of the Post, Stanford University, 2013
\item
  Jared Roehrich, William F. Cody Archive, University of
  Nebraska-Lincoln, 2011--2012
\end{itemize}

\emph{Student Supervision \& Management}

\begin{itemize}
\tightlist
\item
  Alex Ramsey, American Indian Digital History Project, UNO, 2020
\item
  Aaron Burbach, American Indian Digital History Project, UNO, 2020
\item
  Haji Salad, Mobile Community Archives, UNO, 2019-2020
\item
  Sara Ibrahim, Mobile Community Archives, UNO, 2019
\item
  Isis Hernandez-Troche, Mobile Community Archives, UNO, 2018-2019
\item
  Cory Starman, Haphastian Place Naming, UNO, 2018-2019
\end{itemize}

\subsubsection{Directed Readings, Independent Studies, and Creative
Projects}\label{directed-readings-independent-studies-and-creative-projects}

Readings in Digital History, University of Nebraska at Omaha, Cory
Starman, Spring 2018.

Readings in Digital History, University of Nebraska at Omaha, Christine
Taylor, Spring 2018.

\section{Computing Platforms and Digital
Humanities}\label{computing-platforms-and-digital-humanities}

Data analysis: Python, R (programming language), D3.js (visualization),
QGIS / ArcGIS (mapping)

Programming languages: Go, R, Javascript, Python, Ruby, PHP, Unix shell
scripts

Data science: pandas, tidyverse, ggplot, Plotly

Versioning: Git, GitHub, GitLab, SVN

Web Frameworks/Content Management Systems: Django, Vue, Hugo, Omeka \&
Omeka S

Operating Systems: Linux, macOS/iOS, Windows

Database Systems/Persistence: PostgreSQL, SQLite, MySQL

Web/Markup Languages: HTML, CSS, JSON, XML, \LaTeX

Systems Administration: Linux, Apache

Applications: Apple Office Suite, Microsoft Office, Adobe Photoshop,
Final Cut Pro

\section{Languages}\label{languages}

English (native proficiency), Norwegian (bokm\r{a}l) (beginner
proficiency)

\section{Professional Organizations and
Service}\label{professional-organizations-and-service}

\subsubsection{Membership in Professional
Organizations}\label{membership-in-professional-organizations}

American Historical Association\\
Western History Association\\
Southern History Association\\
Organization of American Historians\\
National Council on Public History\\
American Society for Environmental History\\
Agricultural History Society

\subsubsection{Offices in Professional
Organizations}\label{offices-in-professional-organizations}

\emph{Western History Association}

Program Committee, 2018 Annual Meeting, San Antonio, Texas, 2017-2018.

Digital Scholarship Committee, 2011--\emph{present} (Committee chair,
2017--2020)

\vspace{.4cm}

\emph{Alliance of Digital Humanities Organizations}

DHTech Special Interest Group, 2021--\emph{present}

Program reviewer, Digital Humanities Conference 2020, ADHO Conference
Coordinating Committee, Ottawa, Canada, 2020.

Program reviewer, Digital Humanities Conference 2019, ADHO Conference
Coordinating Committee, Utrecht, Netherlands, 2019.

Program reviewer, Digital Humanities Conference 2018, ADHO Conference
Coordinating Committee, Mexico City, Mexico, 2018.

Program reviewer, Digital Humanities Conference 2017, ADHO Conference
Coordinating Committee, Montreal, Canada, 2017.

Libraries and Digital Humanities Special Interest Group, 2017--2020.

\vspace{.4cm}

\emph{Service to the Profession}

Referee for these journals, publishers, grant agencies, and academic
software repositories: \emph{The Programming Historian}, \emph{The
Public Historian}, \emph{Pacific Historical Review}, University of
Michigan Press, National Endowment for the Humanities, Fonds National de
la Recherche.

\vspace{.4cm}

\textbf{Departmental / College / University Service}

\subsubsection{University Committees}\label{university-committees}

\emph{University of Nebraska at Omaha}

Member, University Committee for the Advancement of Teaching,
2020--2021.

Member, University Committee on Technology Resources, Services, and
Planning, 2019--2021.

Member, Sustainability Curriculum Committee, 2019--2021.

\subsubsection{College Committees}\label{college-committees}

\emph{George Mason University}

Member, System Administrator Search Committee, RRCHNM, 2022.

Member, R2 Studios Producer Search Committee, RRCHNM, 2021.

Member, R2 Studios Studio Administrator Search Committee, RRCHNM, 2021.

\vspace{.4cm}

\emph{University of Nebraska at Omaha}

Member, Outreach Archivist Search Committee, UNO Libraries, 2019.

Faculty Chair, Library Faculty, UNO Libraries, 2018--2019.

Faculty Chair-Elect, Library Faculty, UNO Libraries, 2017--2018.

Member, Social Science Librarian Search Committee, UNO Libraries, 2017.

Organizer, Endangered Data Week, UNO Libraries, April 2017.

Member, De-stress Fest Committee, UNO Libraries, 2017.

Member, Triple Bottom Line Committee, UNO Libraries, 2017--2021
(Committee chair, 2017--2019).

\vspace{.4cm}

\emph{Stanford University}

Member, Stanford Interactive Media Group, 2014--2016.

Member, Stanford Interactive Mapping Group, 2013--2016.

Member, Data Visualization Users Group, 2013--2016.

Member, Stanford Mobile Users Group, 2014--2016.

\vspace{.4cm}

\emph{South Dakota State University}

Secretary/Treasurer, Phi Alpha Theta, 2005--2006.

\subsubsection{Department Committees}\label{department-committees}

\emph{University of Nebraska at Omaha}

Member, OER Committee, 2018--2021.

\vspace{.4cm}

\emph{Stanford University}

Member, Digital Publishing Committee, Center for Spatial and Textual
Analysis, 2013--2016.

Member, Digital Humanities Certificate Committee, Center for Spatial and
Textual Analysis, 2013--2016.

Member, Stanford Network Analysis Group, Center for Spatial and Textual
Analysis, 2015--2016.

\vspace{.4cm}

\emph{University of Nebraska-Lincoln}

Member, Digital Outreach Committee, History Graduate Students'
Association, 2012--2013.

Co-Chair, Poster Committee, Center for Great Plains Studies Symposium,
2011--2012.

Vice President, History Graduate Students' Association, 2010--2011.

Assistant Director, Sixth Annual James A. Rawley Conference in the
Humanities, 2010--2011.

Member, Program Committee, James A. Rawley Conference in the Humanities,
2009--2011.

Member, Digital Outreach Committee, History Graduate Students'
Association, 2007--2010.

Executive Director, Fifth Annual James A. Rawley Conference in the
Humanities, 2009--2010.

Chair, ad hoc Website Committee, History Graduate Students' Association,
2008--2009.

Treasurer, History Graduate Students' Association, 2008--2009.

Member, Publicity Committee, Third Annual James A. Rawley Conference in
the Humanities, 2008--2009.

\subsubsection{University Service}\label{university-service}

\emph{George Mason University}

Member, Data Working Group, Roy Rosenzweig Center for History and New
Media, 2021--\emph{present}.

Member, Spatial History Working Group, Roy Rosenzweig Center for History
and New Media, 2021--\emph{present}.

Member, Digital Public History Working Group, Roy Rosenzweig Center for
History and New Media, 2021--\emph{present}.

\vspace{.4cm}

\emph{University of Nebraska at Omaha}

Organizer, Digital Humanities Community of Practice, UNO Libraries,
2018--2021.

Organizer, R User Group, UNO Libraries, 2019--2021.

Faculty Moderator, Student Research and Creative Activity Fair,
University of Nebraska at Omaha, March 2018, March 2019.

Member, Digital Engagement Community of Practice, UNO Libraries,
2017--2018.

\vspace{.4cm}

\emph{Stanford University}

Member, Spotlight Service Team, Center for Interdisciplinary Digital
Research, 2016--2016.

Member, Stanford University Libraries Concierge Advisory Group,
2014--2016.

Member, Working Group on Stanford University Libraries Technical
Infrastructure, 2013--2014.

\vspace{.4cm}

\emph{University of Nebraska-Lincoln}

Member, UNL Libraries Graduate Student Advisory Board, 2010--2012.

\subsubsection{Community Service}\label{community-service}

Member, Environmental History Action Collaborative, Environmental Data
\& Governance Initiative, 2020--\emph{present}.

Deputy voter registrar, Douglas County, Nebraska, certified March 2018.

Member, Mode Shift Omaha, Omaha, Nebraska, 2017--\emph{present}.

Graduate Student Volunteer, History Harvest, Nebraska Public Television,
University of Nebraska-Lincoln, Lincoln, Nebraska, 2010.

Judge, National History Day, Nebraska State Competition, Nebraska
Wesleyan University, Lincoln, Nebraska, 2009.

\subsubsection{Other Professional
Service}\label{other-professional-service}

Member, Committee for Equity and Inclusion, Digital Library Federation,
2020--2021.

Community Representative, Digital Public Library of America, 2017--2018.

Member, Government Records Transparency and Accountability Group,
Digital Library Federation, 2017--2020.
